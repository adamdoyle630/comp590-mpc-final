\documentclass[conference]{IEEEtran}

%the accompanying latexdefs.tex file includes helpful packages and other useful commands. You probably won't need to edit it.
%!TEX root = ./paper.tex

\usepackage[hang,flushmargin]{footmisc}
\usepackage[usenames,dvipsnames]{color}% conflict with acmart
%\usepackage{times}
\usepackage{microtype}
%\usepackage{pdfsync} including this adds a blank page at beginning when using acmart
  \usepackage{amsthm,bm}
  \usepackage{fullpage}
\usepackage{amsmath,amsfonts,amssymb} %hide for acmart
\usepackage{tabu} % detect if table is in math mode
\usepackage{url}
\usepackage{graphicx}
\usepackage{mathtools}
\usepackage{footnote}
\usepackage{framed}
%\usepackage{caption} duplicate
\usepackage{xspace}
\usepackage{multirow}
\usepackage{enumitem}

\usepackage{listings}
\usepackage{color}  % Optional for adding colors

\lstset{
    language=Python,               % set programming language
    basicstyle=\footnotesize\ttfamily, % set font and size
    keywordstyle=\color{blue},     % set keyword color
    commentstyle=\color{gray},     % set comment color
    stringstyle=\color{red},       % set string color
    breaklines=true,               % enable line breaking
    showstringspaces=false         % hide spaces in strings
}


\usepackage{tikz}
\usetikzlibrary{calc,positioning,shapes,shadows,arrows,fit}

\usepackage{adjustbox}
%\usepackage{amssymb} hide for acmart
\usepackage{amsmath}
\usepackage{amsthm}
\usepackage{anyfontsize}
\usepackage{booktabs}
%\usepackage[font={small}]{caption} duplicate
\usepackage{color}
\usepackage{colortbl}
\usepackage{enumitem}
\usepackage{framed}
\PassOptionsToPackage{hyphens}{url}
\usepackage[pdfstartview=FitH,colorlinks,urlcolor=black,linkcolor=black,citecolor=black,pdfpagelabels,bookmarksopen=true]{hyperref} %hide for acmart
\usepackage{letltxmacro}
\usepackage{mathtools}
\usepackage{microtype}
\usepackage{pgfplotstable}
\usepackage{pgfplots}
\usepackage{scalefnt}
\usepackage{setspace}
\usepackage{xcolor}
\usepackage{xspace}
\usepackage{titlesec}
\usepackage{wrapfig}
\usepackage{subcaption}
\usepackage[most]{tcolorbox}
\usepackage{lipsum}
\usepackage{cleveref}

\newcommand{\ignore}[1]{}

\numberwithin{equation}{section}

%fields and groups
\newcommand{\F}{\mathbb{F}}
\newcommand{\Q}{\mathbb{Q}}
\newcommand{\N}{\mathbb{N}}
\newcommand{\Z}{\mathbb{Z}}
\newcommand{\R}{\mathbb{R}}
%\newcommand{\C}{\mathbb{C}} hide for acmart
\newcommand{\Qbar}{\overline{\Q}}
%\newcommand{\G}{\mathbb{G}} hide for acmart
\newcommand{\Vs}{\mathbb{V}}
\newcommand{\Fbar}{\overline{\mathbb{F}}}

\newcommand{\todo}[1]{{\color{red} {\bf TODO}:~{#1}}}
\newcommand{\note}[1]{{\color{blue} {\bf NOTE}:~{#1}}}
\newcommand{\btodo}[1]{{\color{blue} {\bf TODO}:~{#1}}}
\newcommand{\ltodo}[2]{{\color{blue} {\bf TODO (locked by {#1})}:~{#2}}}
\newcommand{\lbtodo}[2]{{\color{blue} {\bf TODO (locked by {#1}}:~{#2}}}

\renewcommand{\paragraph}[1]{\medskip\noindent {\bf {#1}}}


\newcommand{\calA}{\ensuremath{\mathcal{A}}}
\newcommand{\calB}{\ensuremath{\mathcal{B}}}
\newcommand{\calC}{\ensuremath{\mathcal{C}}}
\newcommand{\calD}{\ensuremath{\mathcal{D}}}
\newcommand{\calE}{\ensuremath{\mathcal{E}}}
\newcommand{\calF}{\ensuremath{\mathcal{F}}}
\newcommand{\calG}{\ensuremath{\mathcal{G}}}
\newcommand{\calH}{\ensuremath{\mathcal{H}}}
\newcommand{\calI}{\ensuremath{\mathcal{I}}}
\newcommand{\calJ}{\ensuremath{\mathcal{J}}}
\newcommand{\calK}{\ensuremath{\mathcal{K}}}
\newcommand{\calL}{\ensuremath{\mathcal{L}}}
\newcommand{\calM}{\ensuremath{\mathcal{M}}}
\newcommand{\calN}{\ensuremath{\mathcal{N}}}
\newcommand{\calO}{\ensuremath{\mathcal{O}}}
\newcommand{\calP}{\ensuremath{\mathcal{P}}}
\newcommand{\calQ}{\ensuremath{\mathcal{Q}}}
\newcommand{\calR}{\ensuremath{\mathcal{R}}}
\newcommand{\calS}{\ensuremath{\mathcal{S}}}
\newcommand{\calT}{\ensuremath{\mathcal{T}}}
\newcommand{\calU}{\ensuremath{\mathcal{U}}}
\newcommand{\calV}{\ensuremath{\mathcal{V}}}
\newcommand{\calW}{\ensuremath{\mathcal{W}}}
\newcommand{\calX}{\ensuremath{\mathcal{X}}}
\newcommand{\calY}{\ensuremath{\mathcal{Y}}}
\newcommand{\calZ}{\ensuremath{\mathcal{Z}}}

% THEOREMS %%%%%%%%%%%%%%%%%%%%%%%%%%%%%%%%%%%%%%%%%%%%%%%%%%%%%%%%%%%%%%%%%%%
%
% Theorem definitions

  \theoremstyle{plain} \newtheorem{theorem}{Theorem}[section]
  \newtheorem{lemma}[theorem]{Lemma}
  \newtheorem{claim}[theorem]{Claim}
  \newtheorem{proposition}[theorem]{Proposition}
  \newtheorem{corollary}[theorem]{Corollary}

  \theoremstyle{definition} \newtheorem{defn}[theorem]{Definition}
  \newtheorem{remark}[theorem]{Remark}
  \newtheorem{definition}[theorem]{Definition} \newtheorem{rem}[theorem]{Remark}
  %\newtheorem{alg}[theorem]{Algorithm}
\newtheorem{construction}[theorem]{Construction}
\newtheorem{protocol}[theorem]{Protocol}
\newtheorem{fact}[theorem]{Fact}

\newcommand\numberthis{\addtocounter{equation}{1}\tag{\theequation}}

    \newcommand{\Theorem}[1]{\hyperref[#1]{Theorem~\ref*{#1}}}
    \newcommand{\Lemma}[1]{\hyperref[#1]{Lemma~\ref*{#1}}}
    \newcommand{\Corollary}[1]{\hyperref[#1]{Corollary~\ref*{#1}}}
    \newcommand{\Definition}[1]{\hyperref[#1]{Definition~\ref*{#1}}}
    \newcommand{\Example}[1]{\hyperref[#1]{Example~\ref*{#1}}}
    \newcommand{\Remark}[1]{\hyperref[#1]{Remark~\ref*{#1}}}
    \newcommand{\Fact}[1]{\hyperref[#1]{Fact~\ref*{#1}}}
    \newcommand{\Table}[1]{\hyperref[#1]{Table~\ref*{#1}}}
    \newcommand{\Figure}[1]{\hyperref[#1]{Fig.~\ref*{#1}}}
    \newcommand{\Section}[1]{\hyperref[#1]{Section~\ref*{#1}}}
    \newcommand{\Sections}[1]{\hyperref[#1]{Sections~\ref*{#1}}}
    \newcommand{\Appendix}[1]{\hyperref[#1]{Appendix~\ref*{#1}}}
    \newcommand{\Protocol}[1]{\hyperref[#1]{Protocol~\ref*{#1}}}
    \newcommand{\Equation}[1]{\hyperref[#1]{{(\ref*{#1})}}}

\newcommand{\poly}{\ms{poly}}
\newcommand{\negl}{\ms{negl}}


\newcommand{\sk}{\ms{sk}}
\newcommand{\vk}{\ms{vk}}
\newcommand{\ct}{\ms{ct}}
\newcommand{\hyb}{\ms{Hyb}}

\newcommand{\zostar}{\zo^*}

\newcommand{\rgets}{\mathrel{\mathpalette\rgetscmd\relax}}
\newcommand{\getsr}{\rgets}



\begin{document}

\title{comp590-mpc-final}


\author{\IEEEauthorblockN{Kevin Dong}
\and
\IEEEauthorblockN{Adam Doyle}
\and
\IEEEauthorblockN{Andre Rosado}
\and
\IEEEauthorblockN{Swagat Adhikary}
\and
\IEEEauthorblockN{Prajwal Moharana}
}

\maketitle

\begin{abstract}

This project seeks to implement Multi-Party Computation (MPC) techniques to securely gather and analyze sensitive data about the student body at the University of North Carolina without compromising individual privacy.
Using a combination of frontend and backend technologies, the project will collect data on various metrics such as average academic tenure, financial aid received, family income levels, and GPA, among others. 
The goal is to calculate not only averages but also medians, standard deviations, and to perform linear regression analyses between selected variables. 
By splitting data into encrypted shares that are stored across multiple databases, the system ensures that no single party can access comprehensive personal data, thereby maintaining privacy and security. 
Initial phases have included the development of a responsive frontend user interface and the backend setup for data encryption and distribution. 
This allows for a deep, analytical, yet secure insight into demographic patterns, academic performance, and other crucial metrics, allowing for crucial insights for the university in allocating resources in addressing need areas. 

\end{abstract}


\section{Introduction}

In an era where data privacy has become paramount, educational institutions face significant challenges in collecting and analyzing student data without violating privacy norms and regulations. Multi-Party Computation (MPC) offers a promising solution to this dilemma by allowing multiple parties to compute functions over their inputs while keeping those inputs private. The relevance of MPC in secure data analysis has been extensively documented, with successful applications ranging from secure auctions to privacy-preserving data mining~\cite{MPC1}.

At the University of North Carolina, there is a substantial need to understand the student body's demographics, academic performance, financial needs, and overall well-being without exposing individual data. This project aims to implement an MPC framework to gather and analyze such data securely. Specifically, the framework will handle sensitive information such as GPA, financial aid details, and family income, enabling the university to derive insightful analytics like average study durations, aid distributions, and academic performance indicators across various demographics~\cite{MPC2}.

By leveraging front-end and back-end technologies to encrypt and split data into shares distributed among multiple databases, this project ensures that no single entity can reconstruct or access the entirety of the sensitive data, thus adhering to privacy standards. Such methodologies are aligned with current trends in secure computations and the increasing demand for privacy-preserving analytical techniques in higher education~\cite{PrivacyTrends}.

The implementation of this project not only aims to enhance the strategic planning and resource allocation at the university but also serves as a model for other institutions grappling with similar issues. This introduction sets the stage for a detailed discussion on the project's objectives, the technology stack involved, and the preliminary results obtained from the initial implementation stages.

\section{Solution Overview}

This brief overview of your solution should give enough detail to convince the reader that you know exactly how you're going to solve the technical problems involved in your proposed project.

\section{Progress Update}

Be sure to include the following in your update:
\begin{itemize}
\item What has been accomplished so far.

\item What each team member has done.

\item A GitHub (or other source code repository) link to your project code.
\end{itemize}

Remember that your project update should be 2-3 pages long, excluding references and appendices.

You can also use this template for your final report, but remember that there will be additional sections that you need to add for that part.

\begin{thebibliography}{9}

\bibitem{item1}
Author name. "Title of Paper." Publication venue, year.

\end{thebibliography}

\appendices
\section{Supplementary content}

\todo{Delete this section if there isn't any supplementary material}

\end{document}
